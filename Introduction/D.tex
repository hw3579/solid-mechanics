\section*{Section D}
\label{sec:Section D}
\FloatBarrier % Now figures cannot float above section title


\subsection*{Comparison of results}

By comparing the experimental data with the results of the ANSYS analysis, 
it can be found that all six groups of data measured in the laboratory have 
larger deformation values than the ANSYS analysis. This is due to the fact 
that the modulus of elasticity of the laboratory material is 172.67GPa(Mild Steel) and 63.75GPa(Aluminium), 
both of which are smaller than the theoretical value of the material(210GPa and 70GPa).


In addition, the percentage deformation of mild steel is larger than 
that of aluminium 
(i.e. $\frac{\delta_{AN}-\delta_{FE}}{\delta_{FE}}*100\%$)
and this may also be due to the larger modulus of elasticity of mild steel.

\subsection*{Possible reasons}
$\bullet$ Insufficient material purity or inhomogeneous density 
resulting in a modulus of elasticity less than the theoretical value.

$\bullet$ The memory effect of the metal caused by repeated use of the original 
experimental piece resulted in inaccurate deformation of the measurement.

$\bullet$ Errors caused by machines not being calibrated before measurement or 
by the machine's own measurement issues.

\subsection*{Experimental improvements}

$\bullet$ Replacement of materials with new materials which are not affected by 
the memory effects.

$\bullet$ Choose higher purity and more homogeneous density with higher precision 
components for measurement.

$\bullet$ Change the machine and take several measurements to avoid experimental chance.