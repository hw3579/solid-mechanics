\section*{Section B}
\label{sec:Section B}
\FloatBarrier % Now figures cannot float above section title
In this section, the deformation of each material under different 
forces can be calculated from the data obtained in section A.

\subsection*{Analysis}

We know
\begin{equation} 
    \delta_{max}=\frac{PL^3}{48EI}=\frac{P*0.1^3}{48*E_{exp}*4.5*10^{-11}}
\end{equation}

Using the $E_{exp}$(172.67GPa and 63.75GPa) in different material(Mild Steel and Aluminium)
 with different force(50N, 100N, 150N) in Table \ref{t2}, the data in Table \ref{t3} can be calculated.

\begin{minipage}[htbp]{\textwidth}
    \makeatletter\def\@captype{table}
    \centering
    \scalebox{1.1}{
    \begin{tabular}{lll} 
        \hline
        Bending Displacement  & Mild Steel    & Aluminium     \\ \hline
        $\delta_{AN\_{1}}(P=50N)$ & 0.1341    & 0.3631    \\
        $\delta_{AN\_{2}}(P=100N)$ & 0.2681     & 0.7262   \\
        $\delta_{AN\_{3}}(P=150N)$ & 0.4022     & 1.0893  \\ \hline          
    \end{tabular}} 
    
    (Unit: mm)
    \caption{Experimental results - maximum deformation}
    \label{t3} 
\end{minipage}

\subsection*{Summary}

Bringing the average modulus of elasticity into the equation enables 
a more accurate calculation of the deformation of the material under 
different forces and helps to reduce experimental errors.